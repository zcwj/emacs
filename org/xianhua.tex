\documentclass[11pt]{article}
\usepackage[utf8]{inputenc}
\usepackage[T1]{fontenc}
\usepackage{fixltx2e}
\usepackage{graphicx}
\usepackage{longtable}
\usepackage{float}
\usepackage{wrapfig}
\usepackage{rotating}
\usepackage[normalem]{ulem}
\usepackage{amsmath}
\usepackage{textcomp}
\usepackage{marvosym}
\usepackage{wasysym}
\usepackage{amssymb}
\usepackage{hyperref}
\tolerance=1000
\date{2014-04-27 10:18:57}
\title{闲话}
\hypersetup{
  pdfkeywords={},
  pdfsubject={},
  pdfcreator={}}
\begin{document}

\maketitle
\tableofcontents



\section{闲话21:其人之道,还治其身}
\label{sec-1}
(2010-11-22 14:11:51)
在谈中国的七伤拳之前,先要回答读者一直问的问题,就是小川老兄说的中国的池子是什么?
首先,俺说过最好的蓄水池是你的邻家大院。结果后来看到印度央行的领导们说,印度要达到赶超中国的计划,首先就是要GDP增长速度先赶上。你中国是9\%,俺们就得2位数,是不是?
而要达到这个增长速度,印度需要大批的外来投资。俺们欢迎外国游资。
印度去年的财政年度,外来资金流入量,大概是536亿美元。今天就达到了913亿美元,增长了70\%。因此印度现在大概就吃掉了1千亿美元的货币增量。而印度政府觉得还不够,因此估计这个水池还可以蓄多一些水。
第二个水池,俺本来不想公开说,但是看来大家也会猜到。那就是中国的香港。
香港现在的国际热钱,大概是6500亿港元,换算成美元,大概也有800-900亿美元。因此这里还可以蓄水1千亿到1千5百亿。
当然香港的股市和房价会受到颇大的压力,这就要求香港政府立即动作,保证普通香港市民不要卷入地产和股市泡沫中间,同时增加经济适用房的供应,缓解房屋的硬性需求。
不过如果可以通过香港股市和房地产市场吸收外来资金,然后一举而歼灭之,那么殃及池鱼的香港市民为国家做出了贡献,那么中央政府之后帮他们抹平一些损失,也不是问题。
第三个水池,其实应该是中国的股市和房市。但是中国的房市已经高企,而且大概有50\%的房地产资金,已经是外来的热钱,因此这两个市场的操作空间不大,只能是10\%左右的上上下下震动而已。
把游资吸收一部分,让股市到一个高点,然后就来一个政策发布,用海绵把水挤一把。这个算是古代刑法中凌迟处死的路子。
那么除了股市价格,增大股市的总量,是一个更好的办法。这个办法,就是走当年美国人在互联网泡沫的路子,通过大量的新公司的上升,把外来资金吸收。当年美国的纳斯达克从5000点跌到1千多点,消灭了不少欧洲和中东的资金,算是把美国十年的贸易逆差给平了。
第四个水池,就是中国的中西部发展。让外面的游资走进内陆,去发展中国的新农业、新市镇,从而把热钱,放到冰箱里,把它变成了冷钱。
因此这几个步骤一起来,估计就可以把美联储的6千亿给消化掉。让你变成春雨,来个润物细无声,泥土入海,一去不回。
现在美国针对中国的主要是两个。一个就是人民币汇率,另一个就是关于外贸顺差不应该超过一个经济体GDP的4\%的提议。
对人民币汇率,俺已经说过很多。就是在当前美元独大的国际体系下,无论人民币是涨还是跌,对中国都没有太大好处。因此中国的关注点,不是人民币涨跌的问题,而是人民币变成国际贸易结算货币的问题。
俺一直的提议,就是世界经济天下三分,以人民币、美元和欧元形成一个三角体系。然后各个地方的其他小货币,自己在本地挂靠。
就是说加拿大元、墨西哥比索、巴西里拉等等,你就挂靠在美元上面,英镑、瑞士法郎等等,你就挂靠在欧元上。而日元、韩元、澳元等等,你就挂靠在人民币上。
那么俺们只要确定地区货币,比如说人民币区域内的货币,大家做到货币稳定,那么整个世界的货币体系,取决于三大货币的调整,那么大家就会在保持地区货币的一体性方面,解决当前国际金融和货币市场,盲目强调自由化而引发的大幅度波动,从而导致国际金融次序动荡的问题。
因此人民币是可以缓慢有序的升值的,但是这个升值的时间和速度,必须根据中国和东亚经济的具体情况具体处理。说白了,就是俺大爷高兴就涨一下,不高兴就不涨,还要跌一下。一起取决于大家的表现如何,要想中国动作,就得拿出些实货出来,老想着投机取巧,空手套白狼,那是没门。
而关于贸易顺差不超过4\%,那倒是可以考虑一下。
中国的贸易顺差大,其中一个原因,就是美国对中国实行出口限制和投资限制。你这些限制取消了,自然中美贸易的平衡就会好一些。
当然中国在贸易顺差上需要减少,可是这个比例是贸易顺差和GDP规模的比较。如果俺中国的GDP规模变大,那么这个比例也可以变小嘛,是不是?
俺一直就说,中国的GDP是低估了。低估了多少,是不是应该好好的整理一下。比如说按照PPP的计算,都可以和国际组织展开合作,把中国真正的GDP规模算清楚一点。总而言之,GDP规模变大,也是一个降低比例的方法。
其实这些都不是中国的七伤拳。中国真正的七伤拳,在于放慢经济发展的速度。
目前美国大概2\%的增长,很大程度上在于中国经济的拖动。如果中国主导放慢经济速度,好好做好自己的内部经济结构的调整,说不定就把美国的增长给放没了。而且中国只要开始放慢速度,受到影响的是包括澳洲和加拿大这样的原材料经济,估计可以造成靠套利汇率交易的能源和原材料投机资金,高速平仓。
这种情况下,大量平仓的美元投机资金,暂时会推动美元汇率回升,也会造成发展中国家经济体中出现暂时的资金短缺现象。
而这个时候,中国就应该把手上的美元资产,比如说美国的国债和公司债等等,脱手一些,转而在价格合理的情况下,对这些发展中国家经济进行资金支持,来填满这个空缺。
而且美联储现在在美国受到的质疑非常大,尤其是在货币政策上,和奥巴马政府的高度一致,已经违反了政治中立的原则。这将会受到美国中期选举之后,以财政紧缩作为主要政策诉求的新国会的压力。
如果美联储的举动,在短期内起不到舒缓失业率和稳定房地产市场的作用,那么下一轮的量化宽松,就是所谓你躲过了这一轮洪峰,俺就给你再来一轮的法子,估计难度相当之大。
而如果美国国会决定对中国入口产品进行惩罚性税收制裁,那么中国就要想好应对的法子。
这个就是俺前面说过的,就是中国需要提高对美输出产品的价格。如果美国加入口税50\%,那么中国就应该加上一个同等的数额。这样会导致美国人的生活成本指数翻上去,而中国对美国出口贸易量肯定会大幅降低,但是对加拿大和墨西哥的贸易出口,就会惊人的上升。
这就是俺说过的美国禁酒时期故事的重演。
\end{document}
